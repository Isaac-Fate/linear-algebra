\documentclass[thmcnt=section, color=cyan, 12pt]{my-elegantbook}

% Index page
\usepackage{imakeidx}
\makeindex[columns=2, intoc, options=-s index_style.ist]

% Title and author
\title{Linear Algebra}
\author{Isaac FEI}

% Reference file
\addbibresource{linear-algebra.bib} 

% Image of the book cover
\cover{cover}

\begin{document}

% Print title and cover page
\maketitle

%--------
% Preface
%--------

\frontmatter
\chapter*{Preface}

This book mainly follows the structure of \cite{axlerLinearAlgebraDone1997}.

%------------------------------

% Print table of contents
\tableofcontents
\mainmatter

%-------------------------------
% Main document starts from here
%-------------------------------

%==============================

\chapter{Vector Spaces}

%==============================


\section{Vector Spaces}

Through this book, we use the symbol $\F$ (F for field) to denote $\R$ or $\C$.
Many mathematical definitions arises from simple concrete objects
with certain properties.
The abstraction of \textbf{vector spaces}\index{vector space}
comes from the nice properties possessed by $\F^n$.

\begin{definition} \label{def:1}
	Let $V$ be a set.
	We say $V$ is a vector space over field $F$
	(or we simply say $V$ is a vector space if $\F$ is clear from the context)
	if
	the following properties hold:
	\begin{enumerate}
		\item \textbf{Commutativity} $\mathbf{u} + \mathbf{v} = \mathbf{v} + \mathbf{u}$
		      for all $\mathbf{u}, \mathbf{v} \in V $.
		\item \textbf{Associativity} $(\mathbf{u}+\mathbf{v})+\mathbf{w} = \mathbf{u}+(\mathbf{v}+\mathbf{w})$ for all $\mathbf{u}, \mathbf{v}, \mathbf{w} \in V$.
		\item \textbf{Additive Identity} There exists an element $\mathbf{0} \in V$
		      such that $\mathbf{v} + \mathbf{0} = \mathbf{v}$ for all $\mathbf{v} \in V$.
		      (We only assume the existence of element $\mathbf{0}$,
		      nothing is said about its uniqueness.
		      Although it is indeed unique, we still need to prove this.)
		\item \textbf{Additive Inverse} For every $\mathbf{v} \in V$,
		      there exists an element $\mathbf{w} \in V$ such that $\mathbf{v} + \mathbf{w} = \mathbf{0}$ where $\mathbf{0}$ is the same as the one given above.
		\item \textbf{Multiplicative Identity} $1 \mathbf{v} = \mathbf{v}$ for all $\mathbf{v} \in V$ where $1$ is simply the real number $1$.
		\item \textbf{Distributive Property of Scalar Multiplication over Vector Addition} $a(\mathbf{u} + \mathbf{v}) = a\mathbf{u} + a\mathbf{v}$
		      for all $a \in \F$ and all $\mathbf{u}, \mathbf{v} \in V$.
		\item \textbf{Distributive Property of Scaler Multiplication over Scalar Addition} $(a+b) \mathbf{v} = a\mathbf{v} + b\mathbf{v}$ for all $a, b \in F$ and all $\mathbf{v} \in V$.
	\end{enumerate}
\end{definition}

\begin{note}
	In fact, if a set satisfies property 1 - 4, we say that it is
	an \textbf{Abelian group}\index{Abelian group}
\end{note}

As noted, we need to prove that such element $\mathbf{0}$ is unique.

\begin{proposition}
	Let $V$ be an Abelian group. Suppose both elements $\mathbf{0}$ and $\mathbf{0}^\prime$
	satisfy properties 3 in Definition~\ref{def:1}.
	Then $\mathbf{0} = \mathbf{0}^\prime$.
\end{proposition}

\begin{proof}
	Regarding $\mathbf{0}$ as \textit{an} additive identity, by property 3,
	we have
	\begin{align}
		\mathbf{0}^\prime + \mathbf{0} = \mathbf{0}^\prime
		\label{eq:1}
	\end{align}
	On the other hand, regarding $\mathbf{0}^\prime$ as \textit{an} additive identity,
	we also have
	\begin{align}
		\mathbf{0} + \mathbf{0}^\prime = \mathbf{0}
		\label{eq:2}
	\end{align}
	Because the vector addition is commutative (property 1),
	the left-hand sides of \eqref{eq:1} and \eqref{eq:2} are equal
	and hence $\mathbf{0} = \mathbf{0}^\prime$.
\end{proof}

Now, we can safely say $\mathbf{0}$ is \textit{the} additive identity.
We also say $\mathbf{0}$ is the zero vector.

The next proposition is known as
the \textbf{cancellation property}\index{cancellation property}.

\begin{proposition}[Cancellation Property]
	Let $V$ be an Abelian group.
	Suppose $\mathbf{u}, \mathbf{v}, \mathbf{w} \in V$, we have
	\begin{align*}
		\mathbf{u} + \mathbf{w} = \mathbf{v} + \mathbf{w}
		\implies \mathbf{u} = \mathbf{v}
	\end{align*}
\end{proposition}

\begin{proof}
	By property 4 in Definition~\ref{def:1}, there exists $\mathbf{w}^\prime$
	such that $\mathbf{w} + \mathbf{w}^\prime = \mathbf{0}$.
	It then follows that
	\begin{alignat*}{2}
		 & \quad    &  & \mathbf{u} + \mathbf{w} = \mathbf{v} + \mathbf{w} \\
		 & \implies &  & \mathbf{u} + (\mathbf{w}+ \mathbf{w}^\prime)
		= \mathbf{v} + (\mathbf{w} + \mathbf{w}^\prime)                    \\
		 & \implies &  & \mathbf{u} + \mathbf{0}
		= \mathbf{v} + \mathbf{0}                                          \\
		 & \implies &  & \mathbf{u} = \mathbf{v}
	\end{alignat*}
\end{proof}

The number zero $0$ times any vector is the zero vector $\mathbf{0}$.

\begin{proposition} \label{prop:1}
	Let $V$ be a vector space. Then for any $\mathbf{v} \in V$,
	we have $0 \cdot \mathbf{v} = \mathbf{0}$.
\end{proposition}


The proof is simple.
But we do not skip any intermediate steps to demonstrate
how each property is applied.

\begin{proof}
	Let $\mathbf{v} \in V$ be arbitrary. We have
	\begin{align*}
		\mathbf{v} + 0 \cdot \mathbf{v}
		= 1 \cdot \mathbf{v} + 0 \cdot \mathbf{v}
		= (1 + 0) \cdot \mathbf{v}
		= 1 \cdot \mathbf{v}
		= \mathbf{v}
		= \mathbf{v} + \mathbf{0}
	\end{align*}
	Then applying the cancellation property yields $0 \cdot \mathbf{v} = \mathbf{0}$.
\end{proof}


Any scalar in $\F$ times the zero vector $\mathbf{0}$
is the zero vector itself.

\begin{proposition}
	Let $V$ be a vector space. For any $a \in \F$,
	we have $a \mathbf{0} = \mathbf{0}$.
\end{proposition}

\begin{proof}
	We have
	\begin{align*}
		a\mathbf{0} + \mathbf{0} = a \mathbf{0} = a (\mathbf{0} + \mathbf{0}) = a \mathbf{0} + a \mathbf{0}
	\end{align*}
	Then by the cancellation property, we conclude that $a\mathbf{0} = \mathbf{0}$.
\end{proof}

The element $\mathbf{w}$ in property 4 is also unique
as we shall prove in the following proposition.

\begin{proposition}
	Let $V$ be an Abelian group.
	Pick an element $\mathbf{v} \in V$.
	Suppose both elements $\mathbf{w}$ and $\mathbf{w}^\prime$
	satisfy properties 4 in Definition~\ref{def:1}.
	Then $\mathbf{w} = \mathbf{w}^\prime$.
\end{proposition}

\begin{proof}
	We have
	\begin{align*}
		\mathbf{w}^\prime + \mathbf{0} = \mathbf{w}^\prime
	\end{align*}
	Replacing $\mathbf{0}$ with $\mathbf{v} + \mathbf{w}$ yields
	\begin{alignat*}{2}
		 &          &  & \mathbf{w}^\prime + (\mathbf{v} + \mathbf{w}) = \mathbf{w}        \\
		 & \implies &  & (\mathbf{w}^\prime + \mathbf{v}) + \mathbf{w} = \mathbf{w}^\prime \\
		 & \implies &  & \mathbf{0} + \mathbf{w} = \mathbf{w}^\prime                       \\
		 & \implies &  & \mathbf{w} = \mathbf{w}^\prime
	\end{alignat*}
	where the last equality follows from Proposition~\ref{prop:1}.
\end{proof}

Therefore, for each $\mathbf{v}$, the choice of its
additive inverse $\mathbf{w}$ is unique.
We may then say that such $\mathbf{w}$ is \textit{the} additive inverse of $\mathbf{v}$.
And to make the notation more intuitive, we shall denote
the additive inverse of $\mathbf{v}$ by $-\mathbf{v}$.

The additive inverse $-\mathbf{v}$ of $\mathbf{v}$ can be computed by
the scalar multiplication $-1 \cdot \mathbf{v}$.

\begin{proposition}
	Let $V$ be a vector space and $\mathbf{v} \in V$.
	We have $-\mathbf{v} = -1 \cdot \mathbf{v}$.
\end{proposition}

\begin{proof}
	We have
	\begin{align*}
		\mathbf{v} + (-1 \cdot \mathbf{v}) = 1 \cdot \mathbf{v} + (-1) \cdot \mathbf{v}
		= (1 + (-1)) \mathbf{v}
		= 0 \cdot \mathbf{v}
		= \mathbf{0}
	\end{align*}
	This shows $-1 \cdot \mathbf{v}$ is indeed the additive inverse of $\mathbf{v}$.
\end{proof}

The subtraction operator $-: V \times V \to V$ is defined
as $\mathbf{u} - \mathbf{v} := \mathbf{u} + (-\mathbf{v})$.

%------------------------------

\section{Subspaces}

Let $U$ be a subset of the vector space $V$.
We say that $U$ is a \textbf{vector subspace}\index{vector subspace} of $V$
(or simply subspace of $V$)
if $U$ is also a vector space with the same addition and scalar multiplication
defined on $V$.

To check whether a given subset $U \subseteq V$,
we may simply check that if $U$ contains the zero vector
and if it is closed under
the addition and scalar multiplication.

\begin{proposition}
	Let $U$ be a subset of $V$.
	Then $U$ is a vector subspace of $V$ if and only if
	\begin{enumerate}
		\item $\mathbf{0} \in U$,
		\item $\mathbf{u} + \mathbf{v} \in U$ for all $\mathbf{u}, \mathbf{v} \in U$, and
		\item $a \mathbf{u} \in U$ for all $a \in \F$ and $\mathbf{u} \in U$.
	\end{enumerate}
\end{proposition}

By simple observations,
one may notice that the additive identity in the subspace $U$
is exactly the one in the superspace $V$
and $\mathbf{w}$ is the additive inverse of $\mathbf{u}$ in $U$ if and only if
it is also the additive inverse of $u$ in $V$.

\begin{example}
	$\{\mathbf{0}\}$ and $V$ are subspaces of $V$,
	which are the simplest examples of vector spaces.
\end{example}

\begin{example}
	$\{(x, 0, 0) \mid x \in \F\}$ and $\{(x, y, 0) \mid x, y \in \F\}$
	are subspaces of $\F^3$.
	Specially, when $\F = \R$, this means that
	the 1D $x$-axis line and the
	2D $x$-$y$ plane are subspaces of
	the 3D space.

	In fact, all 1D lines and 2D planes that pass through the origin
	are subspaces of the 3D space.
\end{example}

%------------------------------


\section{Sums and Direct Sums}



%==============================

% References
\printbibliography[heading=bibintoc, title=References]

%==============================

% Print index page
\printindex

\end{document}

