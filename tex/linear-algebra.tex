\documentclass[thmcnt=section, color=cyan, 12pt]{my-elegantbook}

% Index page
\usepackage{imakeidx}
\makeindex[columns=2, intoc, options=-s index_style.ist]

% Title and author
\title{Linear Algebra}
\author{Isaac FEI}

% Reference file
\addbibresource{linear-algebra.bib} 

% Image of the book cover
\cover{cover}

\begin{document}

% Print title and cover page
\maketitle

%--------
% Preface
%--------

\frontmatter
\chapter*{Preface}

This book mainly follows the structure of \cite{axlerLinearAlgebraDone1997}.

%------------------------------

% Print table of contents
\tableofcontents
\mainmatter

%-------------------------------
% Main document starts from here
%-------------------------------

%==============================

\chapter{Vector Spaces}

%==============================


\section{Vector Spaces}

Through this book, we use the symbol $\F$ (F for field) to denote $\R$ or $\C$.
Like many mathematical definitions,
the abstraction of \textbf{vector spaces}\index{vector space}
comes from the nice properties possessed by the concrete mathematical object $\F^n$.

\begin{definition} \label{def:1}
	Let $V$ be a set.
	We say $V$ is a vector space over field $F$
	(or we simply say $V$ is a vector space if $\F$ is clear from the context)
	if
	the following properties hold:
	\begin{enumerate}
		\item \textbf{Commutativity} $\mathbf{u} + \mathbf{v} = \mathbf{v} + \mathbf{u}$
		      for all $\mathbf{u}, \mathbf{v} \in V $.
		\item \textbf{Associativity} $(\mathbf{u}+\mathbf{v})+\mathbf{w} = \mathbf{u}+(\mathbf{v}+\mathbf{w})$ for all $\mathbf{u}, \mathbf{v}, \mathbf{w} \in V$.
		\item \textbf{Additive Identity} There exists an element $\mathbf{0} \in V$
		      such that $\mathbf{v} + \mathbf{0} = \mathbf{v}$ for all $\mathbf{v} \in V$.
		      (We only assume the existence of element $\mathbf{0}$,
		      nothing is said about its uniqueness.
		      Although it is indeed unique, we still need to prove this.)
		\item \textbf{Additive Inverse} For every $\mathbf{v} \in V$,
		      there exists an element $\mathbf{w} \in V$ such that $\mathbf{v} + \mathbf{w} = \mathbf{0}$ where $\mathbf{0}$ is the same as the one given above.
		\item \textbf{Multiplicative Identity} $1 \mathbf{v} = \mathbf{v}$ for all $\mathbf{v} \in V$ where $1$ is simply the real number $1$.
		\item \textbf{Distributive Property of Scalar Multiplication over Vector Addition} $a(\mathbf{u} + \mathbf{v}) = a\mathbf{u} + a\mathbf{v}$
		      for all $a \in \F$ and all $\mathbf{u}, \mathbf{v} \in V$.
		\item \textbf{Distributive Property of Scaler Multiplication over Scalar Addition} $(a+b) \mathbf{v} = a\mathbf{v} + b\mathbf{v}$ for all $a, b \in F$ and all $\mathbf{v} \in V$.
	\end{enumerate}
\end{definition}

\begin{note}
	In fact, if a set satisfies property 1 - 4, we say that it is
	an \textbf{Abelian group}\index{Abelian group}
\end{note}

As noted, we need to prove that such element $\mathbf{0}$ is unique.

\begin{proposition}
	Let $V$ be an Abelian group. Suppose both elements $\mathbf{0}$ and $\mathbf{0}^\prime$
	satisfy properties 3 in Definition~\ref{def:1}.
	Then $\mathbf{0} = \mathbf{0}^\prime$.
\end{proposition}



\begin{proof}
	Regarding $\mathbf{0}$ as \textit{an} additive identity, by property 3,
	we have
	\begin{align}
		\mathbf{0}^\prime + \mathbf{0} = \mathbf{0}^\prime
		\label{eq:1}
	\end{align}
	On the other hand, regarding $\mathbf{0}^\prime$ as \textit{an} additive identity,
	we also have
	\begin{align}
		\mathbf{0} + \mathbf{0}^\prime = \mathbf{0}
		\label{eq:2}
	\end{align}
	Because the vector addition is commutative (property 1),
	the left-hand sides of \eqref{eq:1} and \eqref{eq:2} are equal
	and hence $\mathbf{0} = \mathbf{0}^\prime$.
\end{proof}

Now, we can safely say $\mathbf{0}$ is \textit{the} additive identity.
Moreover, the element $\mathbf{w}$ in property 4 is also unique
as we shall prove in the following proposition.

\begin{proposition}
	Let $V$ be an Abelian group.
	Pick an element $\mathbf{v} \in V$.
	Suppose both elements $\mathbf{w}$ and $\mathbf{w}^\prime$
	satisfy properties 4 in Definition~\ref{def:1}.
	Then $\mathbf{w} = \mathbf{w}^\prime$.
\end{proposition}

\begin{proof}
	We have
	\begin{align*}
		\mathbf{w}^\prime + \mathbf{0} = \mathbf{w}^\prime
	\end{align*}
	Replacing $\mathbf{0}$ with $\mathbf{v} + \mathbf{w}$ yields
	\begin{alignat*}{2}
		 &          &  & \mathbf{w}^\prime + (\mathbf{v} + \mathbf{w}) = \mathbf{w}        \\
		 & \implies &  & (\mathbf{w}^\prime + \mathbf{v}) + \mathbf{w} = \mathbf{w}^\prime \\
		 & \implies &  & \mathbf{0} + \mathbf{w} = \mathbf{w}^\prime                       \\
		 & \implies &  & \mathbf{w} = \mathbf{w}^\prime
	\end{alignat*}
\end{proof}

Therefore, for each $\mathbf{v}$, the choice of its
additive inverse $\mathbf{w}$ is unique.
We may then say that such $\mathbf{w}$ is \textit{the} additive inverse of $\mathbf{v}$.
And to make the notation more intuitive, we shall denote
the additive inverse of $\mathbf{v}$ by $-\mathbf{v}$.

%==============================

% References
\printbibliography[heading=bibintoc, title=References]

%==============================

% Print index page
\printindex

\end{document}

